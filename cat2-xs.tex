% !TeX spellcheck = en_GB
%\documentstyle[12pt]{article}
\documentclass[a4paper,11pt]{article}
\usepackage[bookmarks=true,
              bookmarksnumbered=true, breaklinks=true,
              pdfstartview=FitH, hyperfigures=false,
              plainpages=false, naturalnames=true,
              colorlinks=true,
              pdfpagelabels]{hyperref}
\usepackage{geometry,latexsym,amssymb,amsmath,amsthm,color,bm}
\usepackage[latin5]{inputenc}
\usepackage{enumerate}
\usepackage{enumitem}
\usepackage[T1]{fontenc}
\usepackage{authblk}
%\usepackage{hyperref}
\usepackage[all]{xy}
\usepackage{palatino}
\usepackage{indentfirst}
\usepackage{titlesec}
\usepackage{graphics}
\usepackage{tabularx}
\usepackage{lipsum}
\usepackage{longtable}
\usepackage{array}
\usepackage{fancyvrb,shortvrb}
\usepackage[ruled,vlined]{algorithm2e}
%newcomm.tex,  version 01/05/17

%all newcommands for notes, notesdev and macdw

%% package names and categories, sans serif
\newcommand{\Cat}     {{\sf Cat}}
\newcommand{\GAP}     {{\sf GAP}}
\newcommand{\Gpd}     {{\sf Gpd}}
\newcommand{\SimpSet} {{\sf SimpSet}}
\newcommand{\XMod}    {{\sf XMod}}
\newcommand{\PreXMod} {{\sf PreXMod}}

%% names for categories
\newcommand{\catCat}    {{\bf Cat}}
\newcommand{\cattCat}   {{\bf 2-Cat}}
\newcommand{\catGp}     {{\bf Gp}}
\newcommand{\cattGp}    {{\bf 2-Gp}}
\newcommand{\catGpd}    {{\bf Gpd}}
\newcommand{\cattGpd}   {{\bf 2-Gpd}}
\newcommand{\catGpGpd}  {{\bf GpGpd}}
\newcommand{\catPreXMod}{{\bf PreXMod}}
\newcommand{\catSet}    {{\bf Set}}
\newcommand{\catXComp}  {{\bf XComp}}
\newcommand{\catXMod}   {{\bf XMod}}

%% math Roman
\newcommand{\ab}    {\mathrm{ab}\,}
\newcommand{\arr}   {\mathrm{arr}}
\newcommand{\Act}   {\mathrm{Act}}
\newcommand{\Arr}   {\mathrm{Arr}}
\newcommand{\Aut}   {\mathrm{Aut}}
\newcommand{\coker} {\mathrm{coker}\,}
\newcommand{\cosk}  {\mathrm{cosk}\,}
\newcommand{\Costar}{\mathrm{Costar}}
\newcommand{\Der}   {\mathrm{Der}}
\newcommand{\End}   {\mathrm{End}}
\newcommand{\even}  {\mathrm{even}\,}
\newcommand{\Hol}   {\mathrm{Hol}}
\newcommand{\grp}   {\mathrm{grp}}
\newcommand{\id}    {\mathrm{id}}
\newcommand{\ids}   {\mathrm{ids}}
\newcommand{\im}    {\mathrm{im}\,}
\newcommand{\Inn}   {\mathrm{Inn}}
\newcommand{\mon}   {\mathrm{mon}}
\newcommand{\Mor}   {\mathrm{Mor}}
\newcommand{\nat}   {\mathrm{nat}}
\newcommand{\Ner}   {\mathrm{Ner}}
\newcommand{\obj}   {\mathrm{obj}}
\newcommand{\Ob}    {\mathrm{Ob}}
\newcommand{\odd}   {\mathrm{odd}\,}
\newcommand{\Out}   {\mathrm{Out}}
\newcommand{\Sect}  {\mathrm{Sect}}
\newcommand{\Sing}  {\mathrm{Sing}} 
\newcommand{\sk}    {\mathrm{sk}\,}
\newcommand{\Star}  {\mathrm{Star}}
\newcommand{\Stab}  {\mathrm{Stab}}
\newcommand{\Symm}  {\mathrm{Symm}} 
\newcommand{\Top}   {\mathrm{Top}} 

%% brackets
\newcommand{\dlb} {\ [\hspace{-0.4em}[\ }
\newcommand{\drb} {\ ]\hspace{-0.4em}]\ }

%% blackboard font
\newcommand{\bbA} {\mathbb{A}}
\newcommand{\bbAut} {\mathbb{A}{\mathrm ut}}
\newcommand{\bbB} {\mathbb{B}}
\newcommand{\bbC} {\mathbb{C}}
\newcommand{\bbD} {\mathbb{D}}
\newcommand{\bbE} {\mathbb{E}}
\newcommand{\bbF} {\mathbb{F}}
\newcommand{\bbG} {\mathbb{G}}
\newcommand{\bbH} {\mathbb{H}}
\newcommand{\bbI} {\mathbb{I}}
\newcommand{\bbJ} {\mathbb{J}}
\newcommand{\bbN} {\mathbb{N}}
\newcommand{\bbO} {\mathbb{O}}
\newcommand{\bbQ} {\mathbb{Q}}
\newcommand{\bbR} {\mathbb{R}}
\newcommand{\bbS} {\mathbb{S}}
\newcommand{\bbT} {\mathbb{T}}
\newcommand{\bbZ} {\mathbb{Z}}

%% bold symbol for vectors
\newcommand{\bsyb}  {\boldsymbol{b}}
\newcommand{\bsyc}  {\boldsymbol{c}}
\newcommand{\bsye}  {\boldsymbol{e}}
\newcommand{\bsyg}  {\boldsymbol{g}}
\newcommand{\bsyx}  {\boldsymbol{x}}
\newcommand{\bsyy}  {\boldsymbol{y}}
\newcommand{\bsyz}  {\boldsymbol{z}}

%% calligraphic letters
\newcommand{\calA}{\mathcal{A}}
\newcommand{\calB}{\mathcal{B}}
\newcommand{\calC}{\mathcal{C}}
\newcommand{\calD}{\mathcal{D}}
\newcommand{\calE}{\mathcal{E}}
\newcommand{\calF}{\mathcal{F}}
\newcommand{\calG}{\mathcal{G}}
\newcommand{\calH}{\mathcal{H}}
\newcommand{\calI}{\mathcal{I}}
\newcommand{\calJ}{\mathcal{J}}
\newcommand{\calK}{\mathcal{K}}
\newcommand{\calL}{\mathcal{L}}
\newcommand{\calM}{\mathcal{M}}
\newcommand{\calN}{\mathcal{N}}
\newcommand{\calP}{\mathcal{P}}
\newcommand{\calQ}{\mathcal{Q}}
\newcommand{\calR}{\mathcal{R}}
\newcommand{\calS}{\mathcal{S}}
\newcommand{\calT}{\mathcal{T}}
\newcommand{\calW}{\mathcal{W}}
\newcommand{\calX}{\mathcal{X}}
\newcommand{\calY}{\mathcal{Y}}
\newcommand{\calZ}{\mathcal{Z}}

%%% dotted and barred Greek letters
\newcommand{\da}  {\dot{\alpha}}
\newcommand{\dda} {\ddot{\alpha}}
\newcommand{\bara}{\bar{\alpha}}
\newcommand{\db}  {\dot{\beta}}
\newcommand{\ddb} {\ddot{\beta}}
\newcommand{\barb}{\bar{\beta}}
\newcommand{\dd}  {\dot{\delta}}
\newcommand{\ddd} {\ddot{\delta}}
\newcommand{\bard}{\bar{\delta}}
\newcommand{\dg}  {\dot{\gamma}}
\newcommand{\ddg} {\ddot{\gamma}}
\newcommand{\barg}{\bar{\gamma}}
\newcommand{\ddgo} {\ddot{\gamma}_1}
\newcommand{\ddgt} {\ddot{\gamma}_2}
\newcommand{\dgo} {\dot{\gamma}_1}
\newcommand{\dgt} {\dot{\gamma}_2}
\newcommand{\di}  {\dot{\iota}}
\newcommand{\ddi} {\ddot{\iota}}
\newcommand{\bari}{\bar{\iota}}
\newcommand{\dk}  {\dot{\kappa}}
\newcommand{\ddk} {\ddot{\kappa}}
\newcommand{\bark}{\bar{\kappa}}
\newcommand{\ddl} {\ddot{\lambda}}
\newcommand{\dl} {\dot{\lambda}}
\newcommand{\ddlo} {\ddot{\lambda}_1}
\newcommand{\ddlt} {\ddot{\lambda}_2}
\newcommand{\dlo} {\dot{\lambda}_1}
\newcommand{\dlt} {\dot{\lambda}_2}
\newcommand{\lot} {\lambda_{[2]}}
\newcommand{\barl}{\bar{\lambda}}
\newcommand{\dm}  {\dot{\mu}}
\newcommand{\ddm} {\ddot{\mu}}
\newcommand{\barm}{\bar{\mu}}
\newcommand{\dmo} {\dot{\mu}_1}
\newcommand{\ddmo}{\ddot{\mu}_1}
\newcommand{\dmt} {\dot{\mu}_2}
\newcommand{\ddmt}{\ddot{\mu}_2}
\newcommand{\dn}  {\dot{\nu}}
\newcommand{\ddn} {\ddot{\nu}}
\newcommand{\barn}{\bar{\nu}}
\newcommand{\ddph}{\ddot{\phi}}
\newcommand{\ddpho}{\ddot{\phi}_1}
\newcommand{\ddpht}{\ddot{\phi}_2}
\newcommand{\ddphod}{\ddot{\phi}_1{}^{\prime}}
\newcommand{\ddphtd}{\ddot{\phi}_2{}^{\prime}}
\newcommand{\dph} {\dot{\phi}}
\newcommand{\dpho} {\dot{\phi}_1}
\newcommand{\dpht} {\dot{\phi}_2}
\newcommand{\dphod}{\dot{\phi}_1{}^{\prime}}
\newcommand{\dphtd}{\dot{\phi}_2{}^{\prime}}
\newcommand{\phot} {\phi_{[2]}}
\newcommand{\ddps}{\ddot{\psi}}
\newcommand{\dps} {\dot{\psi}}
\newcommand{\ddch}{\ddot{\chi}}
\newcommand{\dch} {\dot{\chi}}
\newcommand{\barch}{\bar{\chi}}
\newcommand{\dds} {\ddot{\sigma}}
\newcommand{\ds} {\dot{\sigma}}
\newcommand{\ddr} {\ddot{\rho}}
\newcommand{\dr} {\dot{\rho}}
\newcommand{\ddro} {\ddot{\rho}_1}
\newcommand{\ddrt} {\ddot{\rho}_2}
\newcommand{\dro} {\dot{\rho}_1}
\newcommand{\drt} {\dot{\rho}_2}
\newcommand{\ddth} {\ddot{\theta}}
\newcommand{\dth} {\dot{\theta}}
\newcommand{\ddtho} {\ddot{\theta}_1}
\newcommand{\ddtht} {\ddot{\theta}_2}
\newcommand{\dtho} {\dot{\theta}_1}
\newcommand{\dtht} {\dot{\theta}_2}
\newcommand{\thot} {\theta_{[2]}}
\newcommand{\bare} {\bar{\eta}}
\newcommand{\dde} {\ddot{\eta}}
\newcommand{\de} {\dot{\eta}}
\newcommand{\ddx} {\ddot{\xi}}
\newcommand{\dx} {\dot{\xi}}
\newcommand{\ddxo} {\ddot{\xi}_1}
\newcommand{\ddxt} {\ddot{\xi}_2}
\newcommand{\dxo} {\dot{\xi}_1}
\newcommand{\dxt} {\dot{\xi}_2}
\newcommand{\xot} {\xi_{[2]}}
\newcommand{\ddz} {\ddot{\zeta}}
\newcommand{\dz} {\dot{\zeta}}
\newcommand{\ddzo} {\ddot{\zeta}_1}
\newcommand{\ddzt} {\ddot{\zeta}_2}
\newcommand{\dzo} {\dot{\zeta}_1}
\newcommand{\dzt} {\dot{\zeta}_2}
\newcommand{\zot} {\zeta_{[2]}}
\newcommand{\ddD}{\ddot{\Delta}}
\newcommand{\dD}{\dot{\Delta}}

%%% dotted partials
\newcommand{\dbdy}   {\dot{\partial}}
\newcommand{\dbdyo}  {\dot{\partial}_1}
\newcommand{\dbdyt}  {\dot{\partial}_2}
\newcommand{\ddbdy}  {\ddot{\partial}}
\newcommand{\ddbdyo} {\ddot{\partial}_1}
\newcommand{\ddbdyt} {\ddot{\partial}_2}
\newcommand{\bdyot}  {\partial_{[2]}}
\newcommand{\barbdy} {\bar{\partial}}
\newcommand{\barbdyo}{\bar{\partial}_1}
\newcommand{\barbdyt}{\bar{\partial}_2}

%%% dotted Roman letters
\newcommand{\dee}  {\dot{e}}
\newcommand{\deeo} {\dot{e}_1}
\newcommand{\deet} {\dot{e}_2}
\newcommand{\deeod}{\dot{e}_1{}^{\prime}}
\newcommand{\deetd}{\dot{e}_2{}^{\prime}}
\newcommand{\ddee} {\ddot{e}}
\newcommand{\ddeeo}{\ddot{e}_1}
\newcommand{\ddeet}{\ddot{e}_2}
\newcommand{\ddeeod}{\ddot{e}_1{}^{\prime}}
\newcommand{\ddeetd}{\ddot{e}_2{}^{\prime}}
\newcommand{\eeot} {e_{[2]}}
\newcommand{\bareeo}{\bar{e}_1}
\newcommand{\bareet}{\bar{e}_2}
\newcommand{\dhh}  {\dot{h}}
\newcommand{\dhho} {\dot{h}_1}
\newcommand{\dhht} {\dot{h}_2}
\newcommand{\dhhod}{\dot{h}_1{}^{\prime}}
\newcommand{\dhhtd}{\dot{h}_2{}^{\prime}}
\newcommand{\ddhh} {\ddot{h}}
\newcommand{\ddhho}{\ddot{h}_1}
\newcommand{\ddhht}{\ddot{h}_2}
\newcommand{\ddhhod}{\ddot{h}_1{}^{\prime}}
\newcommand{\ddhhtd}{\ddot{h}_2{}^{\prime}}
\newcommand{\hhot} {h_{[2]}}
\newcommand{\barhho}{\bar{h}_1}
\newcommand{\barhht}{\bar{h}_2}
\newcommand{\dii}  {\dot{i}}
\newcommand{\diio} {\dot{i}_1}
\newcommand{\diit} {\dot{i}_2}
\newcommand{\ddii} {\ddot{i}}
\newcommand{\ddiio}{\ddot{i}_1}
\newcommand{\ddiit}{\ddot{i}_2}
\newcommand{\iiot} {i_{[2]}}
\newcommand{\dtt}  {\dot{t}}
\newcommand{\dtto} {\dot{t}_1}
\newcommand{\dttt} {\dot{t}_2}
\newcommand{\dttod}{\dot{t}_1{}^{\prime}}
\newcommand{\dtttd}{\dot{t}_2{}^{\prime}}
\newcommand{\ddtt} {\ddot{t}}
\newcommand{\ddtto}{\ddot{t}_1}
\newcommand{\ddttt}{\ddot{t}_2}
\newcommand{\ddttod}{\ddot{t}_1{}^{\prime}}
\newcommand{\ddtttd}{\ddot{t}_2{}^{\prime}}
\newcommand{\ttot} {t_{[2]}}
\newcommand{\bartto}{\bar{t}_1}
\newcommand{\barttt}{\bar{t}_2}
\newcommand{\duuo} {\dot{u}_1}
\newcommand{\duut} {\dot{u}_2}
\newcommand{\dduu} {\ddot{u}}
\newcommand{\dduuo}{\ddot{u}_1}
\newcommand{\dduut}{\ddot{u}_2}
\newcommand{\uuot} {u_{[2]}}
\newcommand{\baruuo}{\bar{u}_1}
\newcommand{\baruut}{\bar{u}_2}

% dotted capital Roman letters
\newcommand{\dCCo} {\dot{C}_1}
\newcommand{\dCCt} {\dot{C}_2}
\newcommand{\ddCCo}{\ddot{C}_1}
\newcommand{\ddCCt}{\ddot{C}_2}
\newcommand{\CCot} {C_{[2]}}
\newcommand{\CCbt} {C_{\{2\}}}
\newcommand{\CCbo} {C_{\{1\}}}
\newcommand{\CCe} {C_{\emptyset}}
\newcommand{\dGGo} {\dot{G}_1}
\newcommand{\dGGt} {\dot{G}_2}
\newcommand{\ddGGo}{\ddot{G}_1}
\newcommand{\ddGGt}{\ddot{G}_2}
\newcommand{\barGGo}{\bar{\calG}_1}
\newcommand{\barGGt}{\bar{\calG}_2}
\newcommand{\GGot} {G_{[2]}}
\newcommand{\GGbt} {G_{\{2\}}}
\newcommand{\GGbo} {G_{\{1\}}}
\newcommand{\GGe} {G_{\emptyset}}
\newcommand{\dSSo} {\dot{S}_1}
\newcommand{\dSSt} {\dot{S}_2}
\newcommand{\ddSSo}{\ddot{S}_1}
\newcommand{\ddSSt}{\ddot{S}_2}
\newcommand{\SSot} {S_{[2]}}
\newcommand{\SSe} {S_{\emptyset}}

%% dotted calligraphic commands
\newcommand{\dcalC}   {\dot{\calC}}
\newcommand{\dcalCo}  {\dot{\calC}_1}
\newcommand{\dcalCt}  {\dot{\calC}_2}
\newcommand{\ddcalC}  {\ddot{\calC}}
\newcommand{\ddcalCo} {\ddot{\calC}_1}
\newcommand{\ddcalCt} {\ddot{\calC}_2}
\newcommand{\calCot}  {\calC_{[2]}}
\newcommand{\dcalD}   {\dot{\calD}}
\newcommand{\dcalDo}  {\dot{\calD}_1}
\newcommand{\dcalDt}  {\dot{\calD}_2}
\newcommand{\ddcalD}  {\ddot{\calD}}
\newcommand{\ddcalDo} {\ddot{\calD}_1}
\newcommand{\ddcalDt} {\ddot{\calD}_2}
\newcommand{\calDot}  {\calD_{[2]}}
\newcommand{\dcalE}   {\dot{\calE}}
\newcommand{\dcalEo}  {\dot{\calE}_1}
\newcommand{\dcalEt}  {\dot{\calE}_2}
\newcommand{\ddcalE}  {\ddot{\calE}}
\newcommand{\ddcalEo} {\ddot{\calE}_1}
\newcommand{\ddcalEt} {\ddot{\calE}_2}
\newcommand{\calEot}  {\calE_{[2]}}
\newcommand{\dcalG}   {\dot{\calG}}
\newcommand{\dcalGo}  {\dot{\calG}_1}
\newcommand{\dcalGt}  {\dot{\calG}_2}
\newcommand{\ddcalG}  {\ddot{\calG}}
\newcommand{\ddcalGd} {\ddot{\calG}{}^{\prime}}
\newcommand{\ddcalGo} {\ddot{\calG}_1}
\newcommand{\ddcalGt} {\ddot{\calG}_2}
\newcommand{\ddcalGod}{\ddot{\calG}_1{}^{\prime}}
\newcommand{\ddcalGtd}{\ddot{\calG}_2{}^{\prime}}
\newcommand{\calGot}  {\calG_{[2]}}
\newcommand{\dcalM}   {\dot{\calM}}
\newcommand{\dcalMo}  {\dot{\calM}_1}
\newcommand{\dcalMt}  {\dot{\calM}_2}
\newcommand{\ddcalM}  {\ddot{\calM}}
\newcommand{\ddcalMo} {\ddot{\calM}_1}
\newcommand{\ddcalMt} {\ddot{\calM}_2}
\newcommand{\calMot}  {\calM_{[2]}}
\newcommand{\dcalN}   {\dot{\calN}}
\newcommand{\ddcalN}  {\ddot{\calN}}
\newcommand{\dcalR}   {\dot{\calR}}
\newcommand{\dcalRo}  {\dot{\calR}_1}
\newcommand{\dcalRt}  {\dot{\calR}_2}
\newcommand{\ddcalR}  {\ddot{\calR}}
\newcommand{\ddcalRo} {\ddot{\calR}_1}
\newcommand{\ddcalRt} {\ddot{\calR}_2}
\newcommand{\calRot}  {\calR_{[2]}}
\newcommand{\dcalS}   {\dot{\calS}}
\newcommand{\dcalSo}  {\dot{\calS}_1}
\newcommand{\dcalSt}  {\dot{\calS}_2}
\newcommand{\ddcalS}  {\ddot{\calS}}
\newcommand{\ddcalSo} {\ddot{\calS}_1}
\newcommand{\ddcalSt} {\ddot{\calS}_2}
\newcommand{\calSot}  {\calS_{[2]}}

%crossed pairing commands
\newcommand{\ugxh}{G \times H}
\newcommand{\ugth}{G \otimes H}
\newcommand{\gth} {g \otimes h}
\newcommand{\ugtg}{G \otimes G}
\newcommand{\gtg} {g \otimes g}
\newcommand{\ctd} {c \otimes d}
\newcommand{\ugoh}{G \odot H}
\newcommand{\goh} {g \odot h}
\newcommand{\cod} {c \odot d}
\newcommand{\goho}{g_1 \otimes h_1}
\newcommand{\goht}{g_2 \otimes h_2}
\newcommand{\gbh} {g \boxtimes h}
\newcommand{\cbd} {c \boxtimes d}
\newcommand{\gbho}{g_1 \boxtimes h_1}
\newcommand{\gbht}{g_2 \boxtimes h_2}


\newcommand{\asto}    {\;\ast_1\;}
\newcommand{\astt}    {\;\ast_2\;}
\newcommand{\ddasto}  {~\ddot{\ast}_1~}
\newcommand{\ddastt}  {~\ddot{\ast}_2~}
\newcommand{\dasto}   {~\dot{\ast}_1~}
\newcommand{\dastt}   {~\dot{\ast}_2~}

\newcommand{\ot}      {\otimes}
\newcommand{\bt}      {\boxtimes}
\newcommand{\btt}     {\ \tilde{\boxtimes}\ }
\newcommand{\normeq}  {\trianglelefteq}
\newcommand{\sharpz}  {\,\sharp_0\,}
\newcommand{\sharpo}  {\,\sharp_1\,}

%% Gareth's macro for \hash
\def\hashatom{\hbox{{$||$}\hspace*{-0.687em}\raisebox{0.037em}{\rotatebox[origin=c]{90}{$||$}}}}
\def\hashatomscript{\hspace*{0.75mm}\scalebox{0.5}{\hashatom}}
\def\hashatomscriptscript{\hspace*{0.5mm}\scalebox{0.4}{\hashatom}}
\newcommand*{\hashop}{\mathrel{\mathchoice{\hashatom\displaystyle}
{\hashatom\textstyle}
{\hashatomscript\scriptstyle}
{\hashatomscriptscript\scriptscriptstyle}}}
\newcommand{\hash}{\mathop{\hashop}\nolimits}

%\input{tcilatex.tex}

%\titleformat{\section}{\Large\filcenter}{}{1em}{}
%\titleformat{\section}{\Large\bfseries\filcenter}{}{1em}{}
\geometry {textwidth=17cm, textheight=25cm}
\def\baselinestretch{1.1}
%\linespread{1.2}

\renewcommand{\abstractname}{\normalsize\bfseries Abstract}

%\usepackage{setspace}
%\doublespacing
% or:
%\onehalfspacing

\theoremstyle{plain}
\newtheorem{theorem}{Theorem}[section]
\newtheorem{proposition}[theorem]{Proposition}
\newtheorem{lemma}[theorem]{Lemma}
\newtheorem{corollary}[theorem]{Corollary}
\newtheorem{conjecture}[theorem]{Conjecture}
\newenvironment{Prf}{{\bf Proof:} }{\hfill $\Box$\mbox{}}
\theoremstyle{definition}
\newtheorem{definition}[theorem]{Definition}
\newtheorem{example}[theorem]{Example}
\newtheorem{remark}[theorem]{Remark}




\begin{document}
\title{Computing 3-Dimensional Groups : Crossed Squares and  Cat$^2$-Groups}

\author[a]{Z. Arvasi}
\author[a]{A. Odaba\c{s}}
\author[b]{C.~D. Wensley}
\affil[a]{\small{Department of Mathematics and Computer Science, Osmangazi University, Eskisehir, Turkey}}
\affil[b]{\small{School of Computer Science and Electronic Engineering, Bangor University, North Wales, UK}}

\date{}

\maketitle

\begin{abstract}
The category \catXSq\ of crossed squares is equivalent to 
the category \catCatt\ of cat$^2$-groups. 
Functions for computing with these structures have been developed in 
the package \XMod\ written using the \GAP\ computational discrete algebra 
programming language.
This paper includes details of the algorithms used. 
It contains tables listing the $1,086$ \textcolor{red}{(?)} 
isomorphism classes of cat$^2$-groups on groups of order at most $30$. 


\end{abstract}

\noindent{\bf Key Words:} cat$^2$-group, crossed square, \GAP\, \XMod\ 
\\ {\bf Classification:} 18D35, 18G50, .

%---------------------------------------------------------------------------% 
\section{Introduction}

\textcolor{red}
{[This is just a first attempt.]} 

This paper is concerned with the latest contribution to the general programme 
of "computational higher-dimensional group theory" which forms part of the 
"higher-dimensional group theory" programme described, for example, 
by Brown in \cite{brown1}. 

The $2$-dimensional part of these programmes is concerned with group objects 
in the categories of groups or groupoids, and these objects may equivalently 
be considered as crossed modules or cat$^1$-groups. 
Section 2 contains a summary of the definitions of these objects, 
with some examples. 

The first computation part of this programme was described in 
Alp and Wensley \cite{alp2}. 
The output was the package \XMod\ \cite{xmod} for \GAP\ \cite{gap} which, 
at the time, contained functions for constructing crossed modules and cat$^1$-groups of groups, and their morphisms, and conversions from one to another. 
It also contained functions for computing the monoid of derivations of a 
crossed module, and the equivalend monoid of sections of a cat$^1$-group. 
The next development of \XMod\ combined with the \GAP\ package \groupoids\ 
\cite{groupoids} to compute crossed modules of groupoids. 
Later still, a \GAP package \XModAlg\ \cite{xmodalg} was written to compute 
cat$^1$-algebras and crossed modules of algebras. 

The $3$-dimensional part of the higher-dimensional group theory programme 
is concerned with objects in the category \catXSq\ of crossed squares 
and the equivalent cat$^2$-groups category \catCatt. 
The mathematical basis is described in section 3, 
and some computational details are included in section 4. 
In section 5 we enumerate the $1,086$ \textcolor{red}{(?)} isomorphism classes 
of cat$^2$-group structures on the $92$ groups of order at most $30$. 

The impetus for the study of higher-dimensional groups 
comes from algebraic topology -- in the $2$-dimensional case 
by describing the homotopy double groupoid of a based pair of spaces. 
The \XMod\ package follows a purely algebraic approach, 
and does not compute any specifically topological results. 
The interested reader may wish to look at the \GAP\ package
\HAP\ \cite{hap} which also computes with cat$^1$-groups. 



%---------------------------------------------------------------------------% 
\section{Crossed Modules and Cat$^{1}$-Groups}

The notion of crossed module, generalizing the notion of a G-module, 
was introduced by Whitehead \cite{wayted} in the course of his studies on the
algebraic structure of the second relative homotopy group.


\vspace*{4mm}
\textcolor{red}
{[It is confusing having a crossed module $C \to G$ 
and a cat$^1$-group $G \Rightarrow R$. 
So changing all the $C \to G$ to $S \to R$, as in XMod. 
But now $s \in S$ so best not to use $s,t$ as source and target maps 
in the cat$^1$-group, hence changing to tail and head maps $t,h$. 
(I've resisted the temptation to change left actions into right actions!) 
If you are not happy with this, the commit can be ditched.]
} 
\vspace*{4mm}

A \emph{crossed module} consists of a group homomorphism 
$\partial : S \rightarrow R$, endowed with a left action $R$ on $S$ 
(written by $(r,s) \rightarrow {}^{r}s$ for $r \in R$ and $s \in S$) 
satisfying the following conditions:

\begin{center}
	\begin{tabular}{rclll}
	$\partial (^{r}s)$ 
		& $=$ 
			& $r(\partial s)r^{-1}$ 
				&   & $\forall~ s \in S,~ r \in R; $ \\
	$^{(\partial s_{2})}s_{1}$ 
		& $=$ 
			& $s_{2}s_{1}s_{2}^{-1}$ 
				&   & $\forall~ s_{1},s_{2} \in S$. 
	\end{tabular}
\end{center}

The first condition is called the \emph{pre-crossed module property} 
and the second one the \emph{Peiffer identity}. 
We will denote such a crossed module by $\calX = (\partial : S \rightarrow R)$.

A \emph{morphism of crossed modules} 
$(\alpha ,\beta ) : \calX_{1} \rightarrow \calX_{2}$, 
where $\calX_{1} = (\partial_{1} : S_{1} \rightarrow R_{1})$ 
and   $\calX_{2} = (\partial_{2} : S_{2} \rightarrow R_{2})$, 
consists of two group homomorphisms $\alpha : S_{1} \rightarrow S_{2}$
and $\beta : R_{1} \rightarrow R_{2}$ such that 
\[ 
\partial_{2}\circ\alpha ~=~ \beta\circ\partial_{1}, 
\quad \mbox{and} \quad 
\alpha(^{r}s) ~=~ ^{(\beta r)}\alpha s 
\qquad
\forall s \in S,~ r \in R.
\] 

Standard constructions for crossed modules include the following. 
\begin{enumerate}
\item 
A \emph{conjugation crossed module} \index{conjugation crossed module} 
is an inclusion of a normal subgroup $N \unlhd R$, 
where $R$ acts on $N$ by conjugation.
\item 
An \emph{automorphism crossed module} \index{automorphism crossed module} 
has as range a subgroup $R$ of the automorphism group $\Aut(S)$ of $S$ 
which contains the inner automorphism group $\Inn(S)$ of $S$. 
The boundary maps $s \in S$ to the inner automorphism of $S$ by $s$.
\item 
A \emph{zero boundary crossed module} \index{$R$-module} 
has a $R$-module as source and $\partial = 0$.
\item 
Any homomorphism $\partial : S \to R$, with $S$ abelian 
and $\im\partial$ in the centre of $R$, 
provides a crossed module with $R$ acting trivially on $S$.
\item 
A \emph{central extension crossed module} 
\index{central extension crossed module} 
has as boundary a surjection $\partial : S \to R$ with central kernel, 
where $r \in R$ acts on $S$ by conjugation with $\partial^{-1}r$.
\item 
The \emph{direct product} of \index{direct product!of crossed modules} 
$\calX_1 = (\partial_1 : S_1 \to R_1)$ and $\calX_2 = (\partial_2 : S_2 \to R_2)$ 
is $\calX_1 \times \calX_2 
= (\partial_1 \times \partial_2 : S_1 \times S_2 \to R_1 \times R_2)$ 
with $R_1, R_2$ acting trivially on $S_2,\ S_1$ respectively.
\end{enumerate}

Loday reformulated the notion of crossed module as a cat$^{1}$-group. 
Recall from \cite{Loday} that a \emph{cat$^{1}$-group} is a triple $(G,t,h)$ consisting of a group $G$ with two endomorphisms: 
the \emph{tail map} $t$ and the \emph{head map} $h$, 
having a common image $R$ and satisfying the following axioms. 
\begin{equation} \label{cat1-axioms} 
t \circ h = h, \quad  h \circ t = t, 
\quad \mbox{and}\quad  [\ker t,\ker h] = 1. 
\end{equation} 
A \emph{morphism of cat}$^{1}$\emph{-groups} 
$(G_{1},t_1,h_1) \rightarrow (G_{2},t_2,h_2)$ 
is a group homomorphism $f : G_{1} \rightarrow G_{2}$ such that 
\[ 
f \circ t_1 ~=~ t_2 \circ f  
\quad\mbox{and}\quad 
f \circ h_1 ~=~ h_2 \circ f.
\] 
Crossed modules and cat$^{1}$-groups are two-dimensional generalisations 
of a group. 
It was shown in \cite[Lemma 2.2]{Loday} that, 
setting $S = \ker t,~ R = \im t$ and $\partial = h|_{S}$, 
then the conjugation action makes $(\partial : S \rightarrow R)$ 
into a crossed module. 
Conversely, if $(\partial : S \rightarrow R)$ is a crossed module, 
then setting $G = S \rtimes R$ and letting $t,h$ be defined by 
$t(s,r) = (1,r)$ and $h(s,r) = (1,(\partial s)r)$ for $s \in S$, $r \in R$, 
then $(G,t,h)$ is a cat$^{1}$-group.


%---------------------------------------------------------------------------% 
\section{Crossed Squares and Cat$^{2}$-Groups}

The notion of a crossed square is due to Guin-Walery and Loday \cite{walery}. 
A \emph{crossed square of groups} $\calS$ is a commutative square of groups 
\begin{equation} \label{xsq-diag}
\xymatrix@R=20pt@C=20pt
{     &  L \ar[dd]_{\lambda} \ar[rr]^{\kappa} \ar[ddrr]^{\pi} 
         &  & M \ar[dd]^{\mu} 
               &  &  &  L \ar[dd]_{\kappa} \ar[rr]^{\lambda} \ar[ddrr]^{\pi} 
                        &  &  N \ar[dd]^{\nu} \\
\calS \quad = 
      &  &  &  &  &  \tilde{\calS} \quad =  \\  
      &  N \ar[rr]_{\nu} 
         &  & P & &  &  M \ar[rr]_{\mu} 
                        &  &  P } 
\end{equation}
\noindent together with left actions of $P$ on $L,M,N$ 
and a \emph{crossed pairing} map ${\ \bt\ } : M \times N \rightarrow L$. 
Let $M$ act on $N,L$ via $P$ and let $N$ act on $M,L$ via $P$. 
The diagram illustrates a \emph{oriented crossed square} since a choice 
of where to place $M$ and $N$ has been made. 
The \emph{transpose} $\tilde{\calS}$ of $\calS$ is obtained by making the alternative choice. 
Since crossed pairing identities are similar to those for commutators, 
the crossed pairing for $\tilde{\calS}$ is $\btt$ 
where $(n \btt m) = (m \bt n)^{-1}$. 
Transposition gives an equivalence relation on the set of 
oriented crossed squares, and a crossed square is an equivalence class. 

The structure of an oriented crossed square must satisfy the following axioms 
for all $l \in L,~ m,m^{\prime} \in M,~ n,n^{\prime} \in N$ and $p \in P$. 
\begin{enumerate}
\item 
The homomorphisms $\kappa, \lambda, \mu, \nu$ 
and $\pi = \mu\circ\kappa = \nu\circ\lambda$ are crossed modules, 
and both $\kappa, \lambda$ are $P$-equivariant, 
\item
$(mm^{\prime} {\ \bt\ } n) ~=~ (^{m}m^{\prime} {\ \bt\ } {^{m}n})\,(m {\ \bt\ } n)$  
\quad and \quad 
$(m {\ \bt\ } {nn^{\prime}}) ~=~ (m {\ \bt\ } n)\,(^{n}m {\ \bt\ } ^{n}n^{\prime})$, 
\item 
$\kappa(m {\ \bt\ } n) ~=~ m^{n}\ m^{-1}$ 
\quad and \quad 
$\lambda(m {\ \bt\ } n) ~=~ ^{m}n\ n^{-1}$, 
\item 
$(\kappa l {\ \bt\ } n) ~=~ l^{n}\ l^{-1}$  
\quad and \quad 
$(m {\ \bt\ } \lambda l) ~=~ ^{m}l\ l^{-1}$, 
\item 
$^{p}(m {\ \bt\ } n) ~=~ (^{p}m {\ \bt\ } ^{p}n)$. 
\end{enumerate} 

Standard constructions for crossed squares include the following.
\begin{enumerate}
\item 
If $M,N$ are normal subgroups of the group $P$ then the diagram of inclusions
\[
\xymatrix@R=40pt@C=40pt
{ M \cap N \ar[r]^(0.6){} \ar[d]_{}  
	& N \ar[d]^{} \\
	M \ar[r]_{}  
	& P }
\] 
\noindent together with the actions of $P$ on $M,N$ and $M\cap N$ 
given by conjugation, and the commutator map 
\[
[~,~] ~:~ M\times N \rightarrow M\cap N,\quad 
(m,n)\mapsto [m,n] \,=\, mnm^{-1}n^{-1}, 
\] 
is a crossed square. 
We call this an \emph{inclusion crossed square}.
\item 
The diagram
\[
\xymatrix@R=40pt@C=40pt
{ M \ar[r]^{\alpha} \ar[d]_{\alpha} 
	& \Inn\,M \ar[d]^{\iota} \\
	\Inn\,M \ar[r]_{\iota} 
	& \Aut\,M }
\] 
\noindent is a crossed square, 
where $\alpha $ maps $m\in M$ to the inner automorphism%
\[
\beta_{m} : M \rightarrow M,\quad 
m^{\prime}\mapsto m^{-1}m^{\prime}m, 
\]
and where $\iota $ is the inclusion of $\Inn\,M$ in $\Aut\,M$; 
the actions are standard; and the crossed pairing is
\[
\bt ~:~ \Inn\,M \times \Inn\,M \rightarrow M,\quad 
(\beta_{m},\beta_{m^{\prime}}) \;\mapsto\; [m,m^{\prime}]~.
\]
\item 
If $P$ is a group and $M,N$ are ordinary $P$-modules, 
and if $A$ is an arbitrary abelian group on which $P$ is assumed to act trivially, 
then there is a crossed square
\[
\xymatrix@R=40pt@C=40pt
{ A \ar[r]^{0} \ar[d]_{0}  
	& N \ar[d]^{0} \\
	M \ar[r]_{0} 
	& P }
\]
\item 
Given two crossed modules, $(\mu : M \rightarrow P)$ and $(\nu : N \rightarrow P)$, 
there is a universal crossed square defining a tensor product of the crossed modules, 
\[
\xymatrix@R=40pt@C=40pt
{ M \otimes N \ar[d]_{\lambda} \ar[r]^{\kappa} 
	& M \ar[d]^{\mu} \\ 
	N \ar[r]_{\nu} 
	& P } 
\]
\end{enumerate}

A crossed square (\ref{xsq-diag}) can be thought of as a horizontal or vertical 
crossed module of crossed modules:
\[
\xymatrix@R=20pt@C=20pt
{ L \ar[dd]_{\lambda}  
	&  &  M \ar[dd]^{\mu} 
	      &  &  L \ar[rr]^{\kappa}
	            &  {} \ar[dd]^{(\lambda,\mu)} 
	               &  M \\ 
\quad \ar[rr]^{(\kappa,\nu)} 
    &  &  &  &  &  & \quad \\
  N &  &  P 
	      &  &  N \ar[rr]_{\nu} 
	            &  {} 
	               &  P 
} 
\]
\noindent 
where $(\kappa,\nu)$ is the boundary of the crossed module with 
domain $(\lambda : L \rightarrow N)$ and codomain $(\mu : M \rightarrow P)$. 
(See also \cite{wensley_notes} section 10.2.)

There is an evident notion of morphism of crossed squares  
which preserves all the structure, 
and we obtain a category \textbf{Crs}$^{2}$, the category of crossed squares.

Although when first introduced by Loday and Walery \cite{walery} 
the notion of crossed square of groups was not linked to that of cat$^{2}$-groups, 
it was in this form that Loday gave their generalisation 
to an $n$-fold structure, cat$^{n}$-groups (see \cite{Loday}). 
When $n=1$ this is the notion of cat$^1$-group given earlier.

When $n=2$ we obtain a cat$^{2}$-group. 
Again we have a group $G$, but this time with two independent cat$^{1}$-group 
structures on it. 
A \emph{cat$^{2}$-group} is a $5$-tuple, $(G,t_1,h_1,t_2,h_2)$, 
where $(G,t_{i},h_{i}),~ i=1,2$ are cat$^{1}$-groups, and
\[
t_{1}t_{2} ~=~ t_{2}t_{1}, \quad 
h_{1}h_{2} ~=~ h_{2}h_{1}, \quad 
t_{1}h_{2} ~=~ h_{2}t_{1}, \quad
t_{2}h_{1} ~=~ h_{1}t_{2}. 
\]
A morphism of cat$^{2}$-groups is a triple $(\alpha ,\beta ,\gamma)$,
as shown in the diagram diagram
\[
\xymatrix@R=40pt@C=40pt 
{ R_1 \ar[d]_{\beta} 
	& G \ar[d]_{\alpha} \ar@<-0.4ex>[l]_{t_1,h_1} \ar@<+0.4ex>[l]  
	                    \ar@<+0.5ex>[r]^{t_2,h_2} \ar@<-0.4ex>[r] 
		& R_2 \ar[d]^{\gamma} \\
  R'_1 
	& G' \ar@<+0.4ex>[l]^{t'_1,h'_1} \ar@<-0.4ex>[l] 
	     \ar@<-0.4ex>[r]_{t'_2,h'_2} \ar@<+0.4ex>[r] 
		& R'_2 
}
\]
\noindent where 
$\alpha : G \to G^{\prime},~ \beta = \alpha|_{R_1}$ 
and $\gamma = \alpha|_{R_2}$ are homomorphisms satisfying: 
\[ 
\beta \circ t_1 = t_1^{\prime} \circ \alpha, \qquad 
\beta \circ h_1 = h_1^{\prime} \circ \alpha, \qquad 
\gamma \circ t_2 = s_2^{\prime} \circ \alpha, \qquad 
\gamma \circ h_2 = h_2^{\prime} \circ \alpha. 
\] 
We thus obtain a category \textbf{Cat}$^{2}$\textbf{-Grp}, 
the category of cat$^{2}$-groups. 
The following proposition was given by Loday \cite{Loday}. 
We only present the sketch proof (see also \cite{mutpor}) of this result 
as we need some indication of proofs for later use.

\begin{proposition}
\label{loday} \emph{(\cite{Loday})} 
There is an equivalence of categories between the category of cat$^{2}$-groups 
and that of crossed squares.
\end{proposition}
\begin{proof}
The cat$^{2}$-group $(G,t_1,h_1,t_2,h_2)$ determines a diagram of homomorphisms 
\[
\xymatrix@R=50pt@C=50pt
{ \ker t_1 \cap \ker t_2 \ar[d]_{(\id,\partial_2)} \ar[r]^{(\partial_1,\id)} 
  	  & \im t_1 \cap \ker t_2 \ar[d]^{(\id,\partial_2)} \\ 
  \ker t_1 \cap \im t_2 \ar[r]_{(\partial_1,\id)}  
	  & \im t_1 \cap \im t_2 } 
\]
\noindent where each morphism is a crossed module for the natural action, 
conjugation in $G$. 
The required crossed pairing is given by the commutator in $G$ since, 
if $x \in \im t_1 \cap \ker t_2$ and $y \in \ker t_1 \cap \im t_2$ 
then $[x,y] \in \ker t_1 \cap \ker t_2$. 
It is routine to check the crossed square axioms.
	
Conversely, if
\[
\xymatrix@R=40pt@C=40pt
{ L \ar[d]_{\lambda} \ar[r]^{\kappa}  
	  & M \ar[d]^{\mu} \\ 
  N \ar[r]_{\nu} 
	  & P }  
\]
\noindent is a crossed square, 
then we can think of it as a morphism of crossed modules 
$(\kappa,\nu) : (\lambda : L \to N) \rightarrow (\mu : M  \to P)$.
Using the equivalence between crossed modules and cat$^{1}$-groups this
gives a morphism
\[
\partial : (L \rtimes N,t,h) \longrightarrow (M \rtimes P, t^{\prime}, h^{\prime})
\]
of cat$^{1}$-groups. 
There is an action of $(m,p) \in M \rtimes P$ on $(l,n) \in L \rtimes N$ 
given by
\[
^{(m,p)}(l,n) ~=~ (^{m}(^{p}l) (m \bt\ ^{p}n),\ ^{p}n)\,.
\] 
Using this action, we form its associated cat$^{2}$-group with source  
$(L \rtimes N) \rtimes (M \rtimes P)$ 
and induced endomorphisms $t_1,h_1,t_2,h_2$. 
\end{proof}

\begin{definition}
A \emph{cat$^{n}$-group} is a group $G$ with $n$ cat$^{1}$-group structures  
$(G,t_{i},h_{i}),~ 1\leq i\leq n$, such that for all $i \ne j$ 
\[
t_{i}t_{j} = t_{j}t_{i}, \quad 
h_{i}h_{j} = h_{j}h_{i} \quad \mbox{and} \quad 
t_{i}h_{j} = h_{j}t_{i}. 
\]
\end{definition}

A generalisation of crossed square to higher dimensions was given by Ellis
and Stenier (cf. \cite{Sten}). 
It is called a \textquotedblleft crossed $n$-cube \textquotedblright. 
We only use this construction for $n=2$.


%---------------------------------------------------------------------------% 
\section{Computer Implementation}

\GAP\ \cite{gap} is an open-source system for discrete computational
algebra. The system consists of a library of implementations of mathematical
structures: groups, vector spaces, modules, algebras, graphs, codes,
designs, etc.; plus databases of groups of small order, character tables, etc. 
The system has world wide usage in the area of education and scientific research. 
\GAP\ is free software and user contributions to the system are supported. 
These contributions are organized in a form of \GAP\ packages 
and are distributed together with the system.  Contributors can
submit additional packages for inclusion after a reviewing process.

The Small Groups library by Besche, Eick and O'Brien in \cite{bettina} 
provides access to descriptions of the groups of small order. 
The groups are listed up to isomorphism. 
The library contains all groups of order at most 2000 except 1024.

%...............................%
\subsection{2-Dimensional Groups}

The \XMod\ package for \GAP\ contains functions for computing 
crossed modules, cat$^{1}$-groups and their morphisms, 
and was first described in \cite{xmod}. 
This package may be used to select a cat$^{1}$-group from a data file. 
All cat$^{1}$-structures on groups of size up to 70 
(ordered according to the \GAP\ numbering of small groups) 
are stored in a list in the file \texttt{cat1data.g}.

The function \textbf{Cat1Select} may be used in three ways. 
\textbf{Cat1Select( size )} returns the names of the groups with this size, 
while \textbf{Cat1Select( size, gpnum )} prints a list of cat$^1$1-structures 
for this chosen group. 
\textbf{Cat1Select( size, gpnum, num )} returns the chosen cat$^1$1-group.

The following \GAP\ session illustrates the use of these functions.

\begin{Verbatim}[frame=single, fontsize=\small, commandchars=\\\{\}]
\textcolor{blue}{gap> Cat1Select(12);}
Usage:  Cat1Select( size, gpnum, num );
[ "C3 : C4", "C12", "A4", "D12", "C6 x C2" ]
\textcolor{blue}{gap> Cat1Select(12,3);}
Usage:  Cat1Select( size, gpnum, num );
There are 2 cat1-structures for the group A4.
Using small generating set [ f1, f2 ] for source of homs.
[ [range gens], [tail genimages], [head genimages] ] :-
(1)  [ [ f1 ], [ f1, <identity> of ... ], [ f1, <identity> of ... ] ]
(2)  [ [ f1, f2 ],  tail = head = identity mapping ]
2
\textcolor{blue}{gap> C1 := Cat1Select(12,3,2);}
[A4=>A4]
\textcolor{blue}{gap> X1 := XModOfCat1Group(C1);}
[triv->A4]
\end{Verbatim}

A more general notion of cat$^1$-group is implemented in \XMod\, 
where the tail and head maps are no longer required to be endomorphisms on $G$. 
Instead it is required that $tG=hG=R$, and an \emph{embedding} $e : R \to G$ 
is added.  
The axioms in (\ref{cat1-axioms}) then become:  
\[ 
t \circ e \circ h = h, \quad  h \circ e \circ t = t, 
\quad \mbox{and}\quad  [\ker t,\ker h] = 1. 
\] 
We denote such a cat$^1$-group by $(e;t,h : G \to R)$. 


%...............................%
\subsection{3-dimensional Groups}

We have developed new functions for \XMod\ which construct 
(pre-)cat$^{2}$-groups, (pre-)cat$^{3}$-groups, and their morphisms. 
Functions for (pre-)cat$^{2} $-groups include \textbf{PreCat2Group},
\textbf{Cat2Group}, \textbf{IsPreCat2Group}, \textbf{IsCat2Group} 
and \textbf{PreCat2GroupByPreCat1Groups}. 
Attributes of a (pre)cat$^{2}$-group constructed in this way include 
\textbf{GeneratingCat1Groups}, \textbf{Size} and \textbf{Name}. 

As with cat$^1$-groups, we use a more general notion for cat$^2$-groups. 
An \emph{oriented cat$^2$-group} has the form 
\[
\xymatrix@R=40pt@C=60pt
{ G \ar[d]_{t_2,h_2} \ar[r]^{t_1,h_1}  
	  & R_1 \ar[d]^{t_2*e_2*t_1*e_1} \\ 
  R_2 \ar[r]_{t_1*e_1*t_2*e_2} 
	  & R_{12} }  
\]
where $R_1, R_2$ need not be subgroups of $G$, 
but $R_{12}$ is taken to be the common image of 
$t_1*e_1*t_2*h_2$ and $t_2*e_2*t_1*e_1$, a subgroup of $G$. 
\textcolor{red}
{[need to add $e_1,e_2$ to the diagram]} 


Generalizing these functions, we have introduced \textbf{Cat3Group} and \textbf{HigherDimension} which construct cat$^{3}$-groups. 
Functions for general cat$^{n}$-groups will be added as time permits. 
An orientation of a cat$^{3}$-group on $G$ displays a cube whose six faces 
(ordered as front; up, left, right, down,back) are cat$^{2}$-groups. 
The following \GAP\ session illustrates the use of these functions. 
The group $G$ is positioned where the front, up and left faces meet. 

\begin{Verbatim}[frame=single, fontsize=\small, commandchars=\\\{\}]
\textcolor{blue}{gap> a := (1,2,3,4)(5,6,7,8);;}
\textcolor{blue}{gap> b := (1,5)(2,6)(3,7)(4,8);;}
\textcolor{blue}{gap> c := (2,6)(4,8);;}
\textcolor{blue}{gap> G := Group( a, b, c );;}
\textcolor{blue}{gap> SetName( G, "c4c2:c2" );}
\textcolor{blue}{gap> t1a := GroupHomomorphismByImages( G, G, [a,b,c], [(),(),c] );; }
\textcolor{blue}{gap> C1a := PreCat1GroupByEndomorphisms( t1a, t1a );;}
\textcolor{blue}{gap> t1b := GroupHomomorphismByImages( G, G, [a,b,c], [a,(),()] );;}
\textcolor{blue}{gap> C1b := PreCat1GroupByEndomorphisms( t1b, t1b );;}
\textcolor{blue}{gap> C2ab := Cat2Group( C1a, C1b );}
cat2-group with generating (pre-)cat1-groups:
1 : [c4c2:c2 => Group( [ (), (), (2,6)(4,8) ] )]
2 : [c4c2:c2 => Group( [ (1,2,3,4)(5,6,7,8), (), () ] )]
\textcolor{blue}{gap> IsCat2Group( C2ab );}
true
\textcolor{blue}{gap> Size( C2ab );}
[ 16, 2, 4, 1 ]
\textcolor{blue}{gap> t1c := GroupHomomorphismByImages( G, G, [a,b,c], [a,b,c] );;}
\textcolor{blue}{gap> C1c := PreCat1GroupByEndomorphisms( t1c, t1c );;}
\textcolor{blue}{gap> C3abc := Cat3Group( C1a, C1b, C1c );}
cat3-group with generating (pre-)cat1-groups:
1 : [c4c2:c2 => Group( [ (), (), (2,6)(4,8) ] )]
2 : [c4c2:c2 => Group( [ (1,2,3,4)(5,6,7,8), (), () ] )]
3 : [c4c2:c2 => Group( [ (1,2,3,4)(5,6,7,8), (1,5)(2,6)(3,7)(4,8),
(2,6)(4,8) ] )]
\textcolor{blue}{gap> IsPreCat3Group( C3abc );}
true
\textcolor{blue}{gap> HigherDimension( C3abc );}
4
\textcolor{blue}{gap> Front3DimensionalGroup( C3abc );} 
cat2-group with generating (pre-)cat1-groups:
1 : [c4c2:c2 => Group( [ (), (), (2,6)(4,8) ] )]
2 : [c4c2:c2 => Group( [ (1,2,3,4)(5,6,7,8), (), () ] )]
\end{Verbatim}

%............................................%
\subsection{Morphisms of 3-Dimensional Groups}

The function \textbf{MakeHigherDimensionalGroupMorphism} defines morphisms of 
higher dimensional groups, such as cat$^{2}$-groups and crossed squares. 
Functions for cat$^{2}$-group morphisms include 
\textbf{Cat2GroupMorphismByCat1GroupMorphisms}, \textbf{Cat2GroupMorphism} and 
\textbf{IsCat2GroupMorphism}. 
The function \textbf{AllCat2GroupMorphisms} is used to find 
all morphisms between two cat$^{2}$-groups.

In the following \GAP\ session, we obtain a cat$^{2}$-group morphism
using these functions.

\begin{Verbatim}[frame=single, fontsize=\small, commandchars=\\\{\}]
\textcolor{blue}{gap> C2_82 := Cat2Group( Cat1Group(8,2,1), Cat1Group(8,2,2) );}
cat2-group with generating (pre-)cat1-groups:
1 : [C4 x C2 => Group( [ <identity> of ..., <identity> of ...,
<identity> of ... ] )]
2 : [C4 x C2 => Group( [ <identity> of ..., f2 ] )]
\textcolor{blue}{gap> C2_83 := Cat2Group( Cat1Group(8,3,2), Cat1Group(8,3,3) );}
cat2-group with generating (pre-)cat1-groups:
1 : [D8 => Group( [ f1, f1 ] )]
2 : [D8=>D8]
\textcolor{blue}{gap> up1 := GeneratingCat1Groups( C2_82 )[1];;}
\textcolor{blue}{gap> lt1 := GeneratingCat1Groups( C2_82 )[2];;}
\textcolor{blue}{gap> up2 := GeneratingCat1Groups( C2_83 )[1];;}
\textcolor{blue}{gap> lt2 := GeneratingCat1Groups( C2_83 )[2];;}
\textcolor{blue}{gap> G1 := Source(up1);; R1 := Range(up1);; Q1 := Range(lt1);;}
\textcolor{blue}{gap> G2 := Source(up2);; R2 := Range(up2);; Q2 := Range(lt2);;}
\textcolor{blue}{gap> homG := AllHomomorphisms(G1,G2);;}
\textcolor{blue}{gap> homR := AllHomomorphisms(R1,R2);;}
\textcolor{blue}{gap> homQ := AllHomomorphisms(Q1,Q2);;}
\textcolor{blue}{gap> upmor := Cat1GroupMorphism( up1, up2, homG[1], homR[1] );;}
\textcolor{blue}{gap> IsCat1GroupMorphism( upmor );}
true
\textcolor{blue}{gap> ltmor := Cat1GroupMorphism( lt1, lt2, homG[1], homQ[1] );;}
\textcolor{blue}{gap> mor2 := PreCat2GroupMorphism( C2_82, C2_83, upmor, ltmor );}
<mapping: cat2-group with generating (pre-)cat1-groups:
1 : [C4 x C2 => Group( [ <identity> of ..., <identity> of ..., 
  <identity> of ... ] )]
2 : [C4 x C2 => Group( [ <identity> of ..., f2 ] )] -> cat
2-group with generating (pre-)cat1-groups:
1 : [D8 => Group( [ f1, f1 ] )]
2 : [D8=>D8] >
\textcolor{blue}{gap> IsCat2GroupMorphism( mor2 );}
true
\textcolor{blue}{gap> mor8283 := AllCat2GroupMorphisms( C2_82, C2_83 );;}
\textcolor{blue}{gap> Length( mor8283 );}
2
\end{Verbatim}

%\textcolor{red}{Length of mor8283 used to be 2, but now 3!} 

%\textcolor{red}{Do we omit this, correct it, or find a better example?} 

%..............................%
\subsection{Natural Equivalence}

By using the natural equivalence of categories of crossed squares and cat$%
^{2}$-groups, the package includes the functions \textbf{%
	CrossedSquareByCat2Group} and \textbf{Cat2GroupByCrossedSquare} which
constructs crossed squares and cat$^{2}$-groups from the given cat$^{2}$%
-groups and crossed squares, respectively.

The following \GAP\ session illustrates the use of these functions.
The dihedral group $D_{20}$ has two normal subgroups $D_{10}$ 
whose intersection is the cyclic $C_5$.  
We construct the crossed square of normal subgroups, 
and then use the conversion functions to obtain the associated cat$^{2}$-group.

\begin{Verbatim}[frame=single, fontsize=\small, commandchars=\\\{\}]
\textcolor{blue}{gap> d20 := DihedralGroup( IsPermGroup, 20 );;}
\textcolor{blue}{gap> gend20 := GeneratorsOfGroup( d20 ); }
[ (1,2,3,4,5,6,7,8,9,10), (2,10)(3,9)(4,8)(5,7) ]
\textcolor{blue}{gap> p1 := gend20[1];;  p2 := gend20[2];;  p12 := p1*p2; }
(1,10)(2,9)(3,8)(4,7)(5,6)
\textcolor{blue}{gap> d10a := Subgroup( d20, [ p1^2, p2 ] );; }
\textcolor{blue}{gap> d10b := Subgroup( d20, [ p1^2, p12 ] );; }
\textcolor{blue}{gap> c5d := Subgroup( d20, [ p1^2 ] );; }
\textcolor{blue}{gap> SetName( d20, "d20" );  SetName( d10a, "d10a" ); }
\textcolor{blue}{gap> SetName( d10b, "d10b" );  SetName( c5d, "c5d" );  }
\textcolor{blue}{gap> XS1 := CrossedSquareByNormalSubgroups( c5d, d10a, d10b, d20 );  }
[  c5d -> d10a ]
[   |  |   ]
[ d10b -> d20  ]
\textcolor{blue}{gap> IsCrossedSquare(XS1); }
true
\textcolor{blue}{gap> C2G1 := Cat2GroupOfCrossedSquare( XS1 ); }
cat2-group with generating (pre-)cat1-groups:
1 : [((d20 |X d10a) |X (d10b |X c5d))=>(d20 |X d10a)]
2 : [((d20 |X d10a) |X (d10b |X c5d))=>(d20 |X d10b)]
\textcolor{blue}{gap> IsCat2Group( C2G1 ); }
true
\textcolor{blue}{gap> Xab := CrossedSquareOfCat2Group( C2ab ); }
crossed square with crossed modules:
up = [Group( [ (1,5)(2,6)(3,7)(4,8) ] ) -> Group( [ ( 2, 6)( 4, 8) ] )]
left = [Group( [ (1,5)(2,6)(3,7)(4,8) ] ) -> Group(
[ (1,2,3,4)(5,6,7,8), (), () ] )]
right = [Group( [ ( 2, 6)( 4, 8) ] ) -> Group( () )]
down = [Group( [ (1,2,3,4)(5,6,7,8), (), () ] ) -> Group( () )]
\textcolor{blue}{gap> IsCrossedSquare( Xab ); }
true
\textcolor{blue}{gap> IdGroup( Xab ); }
[ [ 2, 1 ], [ 2, 1 ], [ 4, 1 ], [ 1, 1 ] ]
\end{Verbatim}


%---------------------------------------------------------------------------% 
\section{Table of cat$^{2}$-groups}

The new the function \textbf{AllCat2Groups(G)} which constructs all cat$^{2}$%
-groups $(G,s_1,t_1,s_2,t_2)$ over $G$. The function \textbf{%
	AreIsomorphicCat2Groups} is used for checking isomorphism among cat$^{2}$%
-groups. The functions \textbf{IsomorphicCat2GroupFamily} and \textbf{%
	AllCat2GroupsUpToIsomorphism} are used to used to classify given cat$^{2}$%
-groups.

In the following \GAP\ session, we compute all cat$^{2}$-groups on $C_{4}
\times C_{2}$ and their isomorphism families. 
The tenth class consists of cat$^2$-group numbers $[31,34,35,38]$, 
so \texttt{iso82[10]=all82[31]}. 

\begin{Verbatim}[frame=single, fontsize=\small, commandchars=\\\{\}]
\textcolor{blue}{gap> c4c2 := SmallGroup(8,2);;}
\textcolor{blue}{gap> all82 := AllCat2Groups( c4c2 );;}
\textcolor{blue}{gap> Length(all82);}
47
\textcolor{blue}{gap> iso82 := AllCat2GroupsUpToIsomorphism( c4c2 );;}
\textcolor{blue}{gap> Length(iso82);}
14
\textcolor{blue}{gap> AllCat2GroupFamilies( c4c2 );}
[ [ 1 ], [ 2, 5, 8, 11 ], [ 3, 4, 9, 10 ], [ 6, 7, 12, 13 ], 
  [ 14, 17, 22, 25 ], [ 15, 16, 23, 24 ], [ 18, 19, 26, 27 ], 
  [ 20, 21, 28, 29 ], [ 30 ], [ 31, 34, 35, 38 ], [ 32, 33, 36, 37 ], 
  [ 39, 42, 43, 46 ], [ 40, 41, 44, 45 ], [ 47 ] ]
\textcolor{blue}{gap> iso82[10];}
cat2-group with generating (pre-)cat1-groups:
1 : [Group( [ f1, f2, f3 ] ) => Group( [ f2, f2 ] )]
2 : [Group( [ f1, f2, f3 ] ) => Group( [ f2, f1 ] )]
\textcolor{blue}{gap> IsomorphismCat2Groups( all82[31], all82[34] ) = fail;}
false
\end{Verbatim}

In the following table the groups of size at most $30$ are ordered by their
\GAP\ number. 
For each group $G$ we list the number $|IE(G)|$ of idempotent endomorphisms; 
the number $|\calC^1(G)|$ of cat$^1$-groups on $G$, 
followed by the number of isomorphism classes of these; 
and then the number $|\calC^2(G)|$ of cat$^2$-groups on $G$, 
and the number of these isomorphism classes. 
The number of isomorphism classes $\calC^1(G)$ of cat$^{1}$-groups 
is given in \cite{alp2}. 
The size of the set of all cat$^{2}$-groups and the number of isomorphism classes 
of cat$^{2}$-groups on $G$ are computed by given the functions in the manuscript.

We may reduce the size of the table by noting the results for certain simple families. 
When $G=C_{p^{k}}$ is cyclic, with $p$ prime, 
the only idempotent endomorphisms are the identity and zero maps. 

All the cat$^1$-groups have equal tail and head maps, 
and all isomorphism classes are singletons. 
The groups row in the table below lists, for each cat$^2$-group, its four groups. 

%\textcolor{red}{(Check this!)} 

\bigskip
\begin{longtable}{ccccccc}
	\hline\hline
	& $G$ 
	    & $|\mathrm{IE}(G)|$ 
	        & $|\calC^1(G)|$ 
	            & $|\calC^1(G)/\cong |$ 
	                & $|\calC^{2}(G)|$ 
	                    & $|\calC^{2}(G)/\cong |$ \\ 
    \hline
	& $C_{p^{k}}$ 
	    & 2 
	        & 2 
	            & 2 
	                & 3 
	                    & 3 \\ 
	\hline
	\multicolumn{7}{l}{%
		\begin{tabular}{ll}
			groups  & $[G,I,I,I],~ [G,I,G,I],~ [G,G,G,G]$  
		\end{tabular}%
	} \\ 
	\hline
\end{longtable}

\bigskip

When $G$ is cyclic and its order is the product of at most four distinct primes,
we obtain the following, 
where $2 \times [G,I,C_x,I]$ denotes $\{[G,I,C_p,I],[G,I,C_q,I]\}$. 

\bigskip
\begin{longtable}{ccccccc}
	\hline\hline
	& $G$ 
	    & $|\mathrm{IE}(G)|$ 
	        & $|\calC^1(G)|$ 
	            & $|\calC^1(G)/\cong |$ 
	                & $|\calC^{2}(G)|$ 
	                    & $|\calC^{2}(G)/\cong |$ \\ 
	\hline
	& $C_{p^{k}q^{j}}$ 
	    & 4 
	        & 4 
	            & 4 
	                & 10 
	                    & 10 \\ 
	\hline
	\multicolumn{7}{l}{%
		\begin{tabular}{ll}
			groups & $[G,I,I,I],~ [G,I,G,I],~ [G,G,G,G],~ 2 \times [G,I,C_x,I],$ \\
				   & $2 \times [G,C_x,G,C_x],~ 2 \times [G,C_x,C_x,C_x],~ 
				     [G,C_p,C_q,I]$%
		\end{tabular}%
	} \\ 
	\hline
	& $C_{p^{k}q^{j}r^{i}}$ 
	    & 8 
	        & 8 
	            & 8 
	                & 36 
	                    & 36 \\ 
	\hline
	\multicolumn{7}{l}{%
		\begin{tabular}{ll}
			groups & $[G,I,I,I],~ [G,I,G,I],~ [G,G,G,G],~ 3 \times [G,I,C_x,I],$ \\
				   & $3 \times [G,C_x,G,C_x],~ 3 \times [G,C_x,C_x,C_x],~ 
				     3 \times [G,C_x,C_y,I],$ \\
				   & $3 \times [G,I,C_{xy},I],~ 3 \times [G,C_{xy},G,C_{xy}],~ 
				     3 \times [G,C_{xy},C_{xy},C_{xy}],$ \\ 
				   & $6 \times [G,C_x,C_{xy},C_x],~ 3 \times [G,C_z,C_{xy},I],~ 
				     3 \times [G,C_{xz},C_{yz},C_z]$%
		\end{tabular}%
	} \\ 
	\hline
	& $C_{p^{k}q^{j}r^{i}s^{h}}$ 
	    & 16 
	        & 16 
	            & 16 
	                & 136 
	                    & 136 \\ 
	\hline
\end{longtable}

\bigskip

When $G=C_{2p}\times C_{2}$ with $p>2$ and $p$ prime 
(e.g. $C_{6} \times C_{2}, C_{10} \times C_{2}, C_{14} \times C_{2}, \ldots$). 

\vspace*{10mm}
\textcolor{red}
{[We should probably not include the lengthy groups list for this case, 
because, for example, there is more than one idempotent endomorphism 
with image $C_2$, so it is not clear what $[G,I,C_2,I],[G,I,C_2,I]$ means. 
But, for now, just add the missing 32nd entry \texttt{[G,I,G,I]}.]}

\textcolor{blue}
{[OK, You are right]}  

\bigskip
\begin{longtable}{ccccccc}
	\hline\hline
	& $G$ 
	    & $|\mathrm{IE}(G)|$ 
	        & $|\calC^1(G)|$ 
	            & $|\calC^1(G)/\cong |$ 
	                & $|\calC^{2}(G)|$ 
	                    & $|\calC^{2}(G)/\cong |$ \\ 
	\hline
	& $C_{2p}\times C_{2}$ 
	    & 16 
	        & 28 
	            & 8 
	                & 136 
	                    & 32 \\ 
	\hline
	\multicolumn{7}{l}{%
		\begin{tabular}{ll}
			groups & $[G,I,I,I],~ [G,I,C_2,I],~ [G,I,C_2,I],~ [G,C_2,C_2,C_2],~ 
			          [G,C_2,C_2,I],$ \\ 
                   & $[G,I,C_2 \times C_2,I],~ [G,C_2,C_2 \times C_2,C_2],~ 
                      [G,C_2,C_2 \times C_2,C_2],$ \\ 
                   & $[G,C_2 \times C_2,C_2 \times C_2,C_2 \times C_2],~ 
                      [G,I,C_p,I],~ [G,C_2,C_p,I],$ \\ 
                   & $[G,C_2,C_p,I],~ [G,C_2 \times C_2,C_p,I],~ [G,C_p,C_p,C_p],~ 
                      [G,I,C_{2p},I],$ \\ 
                   & $[G,I,C_{2p},I],~ [G,C_2,C_{2p},C_2],~ [G,C_2,C_{2p},I],~ 
                      [G,C_2 \times C_2,C_{2p},C_2],$ \\ 
                   & $[G,C_2 \times C_2,C_{2p},C_2],~ [G,C_p,C_{2p},C_p],~ 
                      [G,C_p,C_{2p},C_p],$ \\ 
                   & $[G,C_{2p},C_{2p},C_{2p}],~ [G,C_{2p},C_{2p},C_p],~ 
                      [G,C_2 \times C_2,C_{2p} \times C_2,C_2 \times C_2],$ \\ 
                   & $[G,C_2,C_{2p} \times C_2,C_2],~ 
                      [G,C_2,C_{2p} \times C_2,C_2],~ 
                      [G,C_p,C_{2p} \times C_2,C_p],$ \\ 
                   & $[G,C_{2p},C_{2p} \times C_2,C_{2p}],~ 
                      [G,C_{2p},C_{2p} \times C_2,C_{2p}],~ [G,I,G,I],~ [G,G,G,G]$ 
		\end{tabular}%
	} \\ 
	\hline
\end{longtable}

The following table contains results for those $G$ which do not belong 
to one of these families. 
%{\small \
\begin{longtable}{ccrrrrr}
	\hline\hline
	{\GAP\ }\# 
	    & $G$ 
	        & $|\mathrm{IE}(G)|$ 
	            & $|\calC^1(G)|$ 
	                & $|\calC^1(G)/\cong |$ 
	                    & $|\calC^{2}(G)|$ 
	                        & $|\calC^{2}(G)/\cong |$  \\ 
	\hline
	1/1 & I & 1 & 1 & 1 & 1 & 1 \\ 
	4/2 & K4 & 8 & 14 & 4 & 36 & 9 \\ 
	6/1 & S3 & 5 & 4 & 2 & 7 & 3 \\ 
	8/2 & C4 $\times$ C2 & 10 & 18 & 6 & 47 & 14 \\ 
	8/3 & D8 & 10 & 9 & 3 & 17 & 5 \\ 
	8/4 & Q8 & 2 & 1 & 1 & 1 & 1 \\ 
	8/5 & C2 $\times$ C2 $\times$ C2 & 58 & 226 & 6 & 1,711 & 23 \\ 
	9/2 & C3 $\times$ C3 & 14 & 38 & 4 & 93 & 9 \\ 
	10/1 & D10 & 7 & 6 & 2 & 11 & 3 \\ 
	12/1 & C3 $\ltimes$ C4 & 5 & 4 & 2 & 7 & 3 \\ 
	12/3 & A4 & 6 & 5 & 2 & 9 & 3 \\ 
	12/4 & D12 & 21 & 12 & 4 & 41 & 10 \\ 
	14/1 & D14 & 9 & 8 & 2 & 15 & 3 \\ 
	16/2 & C4 $\times$ C4 & 26 & 98 & 5 & 231 & 11 \\ 
	16/3 & (C4 $\times$ C2) $\ltimes$ C2 & 18 & 25 & 4 & 41 & 6 \\ 
	16/4 & C4 $\ltimes$ C4 & 10 & 17 & 3 & 25 & 4 \\ 
	16/5 & C8 $\times$ C2 & 10 & 18 & 6 & 47 & 14 \\ 
	16/6 & C8 $\ltimes$ C2 & 6 & 5 & 2 & 9 & 3 \\ 
	16/7 & D16 & 18 & 9 & 2 & 17 & 3 \\ 
	16/8 & QD16 & 10 & 5 & 2 & 9 & 3 \\ 
	16/9 & Q16 & 2 & 1 & 1 & 1 & 1 \\ 
	16/10 & C4 $\times$ C2 $\times$ C2 & 82 & 322 & 12 & 2,875 & 53 \\ 
	16/11 & C2 $\times$ D8 & 82 & 97 & 9 & 473 & 24 \\ 
	16/12 & C2 $\times$ Q8 & 18 & 17 & 3 & 25 & 4 \\ 
	16/13 & (C4 $\times$ C2) $\ltimes$ C2 & 26 & 13 & 2 & 25 & 3 \\ 
	16/14 & C2 $\times$ C2 $\times$ C2 $\times$ C2 
	          & 382 & 4,162 & 9 & 298,483 & 53  \\ 
	18/1 & D18 & 11 & 10 & 2 & 19 & 4 \\ 
	18/3 & C3 $\times$ S3 & 12 & 8 & 4 & 24 & 10 \\ 
	18/4 & (C3 $\times$ C3) $\ltimes$ C2 & 47 & 118 & 4 & 541 & 9 \\ 
	18/5 & C6 $\times$ C3 & 28 & 76 & 8 & 358 & 32 \\ 
	20/1 & Q20 & 7 & 6 & 2 & 11 & 3 \\ 
	20/3 & C4 $\ltimes$ C5 & 7 & 6 & 2 & 11 & 3 \\ 
	20/4 & D20 & 31 & 18 & 4 & 65 & 10 \\ 
	21/1 & C3 $\ltimes$ C7 & 9 & 8 & 2 & 15 & 3 \\ 
	22/1 & D22 & 13 & 12 & 2 & 23 & 3 \\ 
	24/1 & C3$\ltimes$C8 & 5 & 4 & 2 & 7 & 3 \\ 
	24/3 & SL(2,3) & 6 & 1 & 1 & 1 & 1 \\ 
	24/4 & Q24 & 5 & 4 & 2 & 7 & 3 \\ 
	24/5 & S3 $\times$ C4 & 27 & 12 & 4 & 41 & 10 \\ 
	24/6 & D24 & 33 & 20 & 4 & 75 & 10 \\ 
	24/7 & Q12 $\times$ C2 & 25 & 36 & 6 & 115 & 14 \\ 
	24/8 & D8 $\ltimes$ C3 & 23 & 12 & 4 & 41 & 10 \\ 
	24/9 & C12 $\times$ C2 & 20 & 36 & 12 & 178 & 52 \\ 
	24/10 & D8 $\times$ C3 & 20 & 18 & 6 & 59 & 17 \\ 
	24/11 & Q8 $\times$ C3 & 4 & 2 & 2 & 3 & 3 \\ 
	24/12 & S4 & 12 & 5 & 2 & 9 & 3 \\ 
	24/13 & A4 $\times$ C2 & 15 & 10 & 4 & 31 & 10 \\ 
	24/14 & S3 $\times$ K4 & 157 & 116 & 8 & 999 & 32 \\ 
	24/15 & C6 $\ltimes$ K4 & 116 & 452 & 12 & 6,786 & 88 \\ 
	25/2 & C5 $\times$ C5 & 32 & 152 & 4 & 348 & 9 \\ 
	26/1 & D26 & 15 & 14 & 2 & 27 & 3 \\ 
	27/2 & C9 $\times$ C3 & 20 & 56 & 6 & 138 & 14 \\ 
	27/3 & (C3 $\times$ C3) $\ltimes$ C3 & 38 & 37 & 2 & 73 & 3 \\ 
	27/4 & C9 $\ltimes$C3 & 11 & 10 & 2 & 19 & 3 \\ 
	27/5 & C3 $\times$ C3 $\times$ C3 & 236 & 2,108 & 6 & 24,222  & 16 \\ 
	28/1 & Q28 & 9 & 8 & 2 & 15 & 3 \\ 
	28/3 & D28 & 41 & 24 & 4 & 89 & 10 \\ 
	30/1 & D6 $\times$ C5 & 10 & 8 & 4 & 24 & 10 \\ 
	30/2 & D10 $\times$ C3 & 14 & 12 & 4 & 38 & 10 \\ 
	30/3 & D30 & 25 & 24 & 4 & 92 & 10 \\ \hline
\end{longtable}
%}


%---------------------------------------------------------------------------% 
\section*{Acknowledgement}

The first and second authors were supported by Eskisehir Osmangazi
University Scientific Research Projects (Grant No: 2017/19033).


\vspace*{10mm}
\noindent
\textcolor{red}
{[What is the correct bibliography convention? 
  Are not the \emph{titles} emphasised, and not the Journal name?]} 

\begin{thebibliography}{99}
	\bibitem{alp2} \textsc{Alp M. and Wensley C.D.}, 
	Enumeration of cat$^{1}$-groups of low order, 
	\emph{Int. J. Algebra Comput.}, 10, 407-424, (2000).
	
	\bibitem{Artin} \textsc{M. Artin} and \textsc{B. Mazur}, 
	\textrm{On the Van Kampen theorem, } 
	\emph{Topology,} \textbf{5}, 179-189, (1966).
	
	\bibitem{arvasi2} \textsc{Arvasi Z. and Ulualan E.}, 
	On Algebraic Models for Homotopy 3-types, 
	\emph{Journal of Homotopy and Related Structures}, 1, 1-27, (2006).
	
	\bibitem{xmodalg} \textsc{Arvasi Z. and Odabas A.},  
	Crossed Modules and Cat$^1$-algebras, 
	{(manual for the \textsf{XModAlg} share package for \GAP, version 1.12, 
	(2015).)} 
	
	\bibitem{odabas} \textsc{Arvasi Z. and Odabas A.}, 
	Computing 2-dimensional algebras: Crossed modules and Cat1-algebras, 
	\emph{J. Algebra Appl.}, 15, 1650185 (2016).
	
	\bibitem{gap} \textsc{The GAP Group},
	GAP -- Groups, Algorithms, and Programming, Version 4.10.2; (2019). 
	(https://www.gap-system.org).
	
	
	%\bibitem{baus1} \textsc{H.J. Baues}, 
	%\textrm{Algebraic homotopy, } 
	%\emph{Cambridge Studies in Advanced Mathematics, } 
	%\textbf{15}, (1998), 450 pages.
	
	%\bibitem{baus2} \textsc{H.J. Baues}, 
	%\textrm{Combinatorial homotopy and 4-dimenional complexes,} 
	%\emph{Walter de Gruyter, }\textbf{15}, (1991).
	
	%\bibitem{berger} \textsc{C. Berger}, 
	%\textrm{Double Loop spaces, braided monoidal categories 
	%and algebraic 3-types of spaces, } 
	%\emph{Contemporary Mathematics,} \textbf{227}, 46-66, (1999).
	
	\bibitem{Brown} \textsc{R. Brown} and \textsc{J.-L. Loday}, 
	\textrm{Van Kampen theorems for diagram of spaces,} 
	\emph{Topology,} \textbf{26}, 311-335, (1987).
	
	\bibitem{brown1} \textsc{Brown R.}, 
	Higher Dimensional Group Theory, Low Dimensional Topology, 
	\emph{London Math. Soc. Lecture Note Series}, 48, 215-238, (1982).
	
	%\bibitem{Spencer} \textsc{R. Brown} and \textsc{C.B. Spencer}, 
	%\textrm{G-groupoids, crossed modules and the fundamental groupoid 
	%of topological group,} 
	%\emph{Proc. Kon. Ned. Akad.} \textbf{79}, 296-302, (1976).
	
	\bibitem{rb-ng} \textsc{R. Brown} and \textsc{N.D. Gilbert}, 
	\textrm{Algebraic models of 3--types and automorphism structures 
	for crossed modules}, 
	\emph{Proc. London Math. Soc.} \ (3) \textbf{59}, 51-73, (1989).
	
	\bibitem{bettina} \textsc{Besche H.U., Eick B. and O'Brien E.A.}, 
	A millennium project: constructing Small Groups, 
	\emph{\ Internat. J. Algebra Comput.}, 12, 623 - 644 (2002).
	
	%\bibitem{bullejos} \textsc{M.Bullejos, J.G. Cabello} and \textsc{E. Faro},
	%\textrm{On the equivariant 2-type of a G-space}, 
	%\emph{J. Pure Applied Algebra,}\ \textbf{129}, 215-245, (1998).
	
	%\bibitem{cc} \textsc{P. Carrasco} and \textsc{A.M. Cegarra}, 
	%\textrm{Group-theoretic algebraic models for homotopy types}, \ 
	%\emph{J. Pure Applied Algebra,} \ \textbf{75}, 195-235, (1991).
	
	%\bibitem{Conduce} \textsc{D. Conduch\'{e} }, 
	%\textrm{Modules crois{\'{e}}s g{\'{e}}n{\'{e}}ralis{\'{e}}s de longueur 2,} 
	%\emph{J. Pure and Applied Algebra, } \textbf{34}, 155-178, (1984).
	
	%\bibitem{Con1} \textsc{D. Conduch\'{e} }, 
	%\textrm{Simplicial Crossed Modules and Mapping Cones}, 
	%\emph{Georgian Mathematical Journal}, \textbf{10}, No.4, 623-636, (2003).
	
	\bibitem{Sten} \textsc{G.J. Ellis } and \textsc{R. Steiner}, 
	\textrm{Higher dimensional crossed modules and the homotopy groups 
	of (n+1)-ads.}, \ 
	\emph{\ J. Pure and Applied Algebra}, \textbf{46}, 117-136, (1987).
	
	\bibitem{hap} \textsc{G.J. Ellis }, 
	\textrm{Homological Algebra Programming}, \ 
	{(manual for the \HAP\ package for \textsf{GAP}, version 1.19, (2019))}. 

	\bibitem{Ellis} \textsc{G.J. Ellis }, 
	\textrm{Crossed Squares and Combinatorial Homotopy}, \ 
	\emph{Math. Z.}, \textbf{461}, 93-110, (1993).
	
	\bibitem{Loday} \textsc{J.L. Loday}, 
	\textrm{Spaces with finitely many non-trivial homotopy groups }, \ 
	\emph{J. Pure and Applied Algebra}, \textbf{24}, 179-202, (1982).
	
	\bibitem{groupoids} \textsc{Moore E.J.} and \textsc{Wensley C.D.}, 
	Calculations with finite groupoids and their homomorphisms 
	{(manual for the \groupoids\ package for \textsf{GAP}, version 1.68,  
	(2019))}. 

	\bibitem{porter1} \textsc{Porter T.}, 
	The Crossed Menagerie: an introduction to crossed gadgetry and cohomology 
	in algebra and topology. 
	\emph{Lecture Notes}, (2010).
	
	\bibitem{m-w} \textsc{S. Mac Lane} and \textsc{J.H.C. Whitehead}, 
	\textrm{On the 3-type of a complex }, 
	\emph{Proc. Nat. Acad. Sci. U.S.A.,} Vol. \textbf{\ 37}, 41-48, (1950).
	
	\bibitem{Mutlu} \textsc{A. Mutlu} and \textsc{T. Porter}, 
	\textrm{Applications of Peiffer pairings in the Moore complex 
	of a simplicial group}, 
	\emph{Theory and Applications of Categories},  
	Vol.4, No.\textbf{7} 148-173, (1998).
	
	\bibitem{mutpor} \textsc{A. Mutlu} and \textsc{T. Porter}, 
	\textrm{Crossed squares and 2-crossed modules}, 
	\emph{Preprint}, (2003).
	
	\bibitem{Porter} \textsc{T. Porter}, 
	\textrm{n-type of simplicial groups and crossed n-cubes}, 
	\emph{Topology ,} \textbf{32}, 5-24, (1993).
	
	\bibitem{walery} \textsc{D. Guin-Walery} and \textsc{J.L. Loday}, 
	\textrm{Obstruction {\'{a}} l'excision en K-theories alg{\'{e}}brique, } \ 
	\emph{In: Friedlander, E.M.,Stein, M.R.(eds.) 
	Evanston conf. on algebraic K-Theory 1980.} 
	(Lect. Notes Math., vol.854, pp 179-216) Berlin Heidelberg New York:Springer 
	(1981).
	
	\bibitem{wayted} \textsc{J.H.C. Whitehead}, 
	\textrm{Combinatorial homotopy II }, 
	\emph{Bull. Amer. Math. Soc.} \textbf{55}, 453-496, (1949).
	
	\bibitem{wensley_notes} \textsc{Wensley C.D.},  
	Notes on higher dimensional groups and related topics, 
	\emph{Development Version}, (2017).
	
	\bibitem{xmod} \textsc{Wensley C.D., Alp M., Odabas A.} and \textsc{Uslu E.O},  
	\ Crossed modules and cat$^{1}$-groups, 
	{(manual for the \XMod\ package for \textsf{GAP}, version 2.74 (2019))}. 
	
\end{thebibliography}







\end{document}
